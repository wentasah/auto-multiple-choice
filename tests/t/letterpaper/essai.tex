\documentclass[letterpaper]{article}
\usepackage{automultiplechoice}
\begin{document}

\element{q3C2L}{
\begin{question}{q3C2L1}
  \bareme{b=2}
  Voil\`a une premi\`ere question\hfill[2pts]
    \begin{reponses}%
      \mauvaise{1}
      \bonne{2}
      \mauvaise{3}
      \mauvaise{4}
      \mauvaise{5}
      \mauvaise{6}
    \end{reponses}
\end{question}
}%

\element{q3C2L}{
\begin{questionmult}{q3C2L2}
  \bareme{haut=4,p=0,v=0}
  Voil\`a la deuxi\`eme question\hfill[4pts]
    \begin{reponses}
      \bonne{$A$}
      \mauvaise{$B$}
      \mauvaise{$C$}
      \bonne{$D$}
      \bonne{$E$}
      \bonne{$F$}
    \end{reponses}
\end{questionmult}
}

\exemplaire{5}{%

%============================
\noindent
\textsc{\'Etablissement} \hfill \textsc{S3I - Classe} 2014-2015

\vspace*{10pt}
\begin{minipage}{.4\linewidth}
\centering\large\bf \'Evaluation rapide
\end{minipage}
\champnom{%---------------
\fbox{%
\begin{minipage}{.5\linewidth}
Nom et pr\'enom :

\vspace*{.5cm}\dotfill
\vspace*{1mm}
\end{minipage}
}%--- fbox
}%--- champnom

\begin{center}\em
Dur\'ee : 20 minutes

Aucun document n'est autoris\'e.

L'usage de la calculatrice est interdit.

\medskip
\textsc{Utiliser un STYLO NOIR ou un FEUTRE NOIR pour remplir les cases choisies.}

\medskip
Les questions faisant appara\^\i{}tre le symbole \multiSymbole{} peuvent pr\'esenter une ou plusieurs bonnes r\'eponses. 

Les questions sans symbole ont une unique bonne r\'eponse.
\end{center}

\melangegroupe{q3C2L}
\restituegroupe{q3C2L}

}%--- Exemplaires

\end{document}
