\documentclass[12pt,a4paper]{article}

\usepackage{csvsimple}%

\usepackage[francais,bloc,ensemble,completemulti]{automultiplechoice}

%%%%commande création sujet
\newcommand{\sujet}{%
%
\exemplaire{1}{%

%%% debut de l'en-tête des copies :  
\begin{center}
\noindent{}\fbox{\vspace*{3mm}
         \Large\bf\nom{}~\prenom{}\normalsize{}% 
          \vspace*{3mm}
      }
\end{center}

\noindent{\bf QCM  \hfill TEST}

\vspace*{.5cm}
\begin{minipage}{.4\linewidth}
  \centering\large\bf Test\\ Examen du 01/01/2008
\end{minipage}

\begin{center}\em
Durée : 10 minutes.

  Aucun document n'est autorisé.
  L'usage de la calculatrice est interdit.

  Les questions faisant apparaître le symbole \multiSymbole{} peuvent
  présenter zéro, une ou plusieurs bonnes réponses. Les autres ont
  une unique bonne réponse.

  Des points négatifs pourront être affectés à de \emph{très
    mauvaises} réponses.
\end{center}
\vspace{1ex}
%%% fin de l'en-tête

\restituegroupe{general}

\clearpage

%\AMCcleardoublepage   

% \AMCaddpagesto{3}




%%% début de l'en-tête de la feuille de réponses

\begin{center}

{\large\bf Feuille de réponses :}

\noindent{}\champnom{\fbox{%
         \Large\bf\nom{}~\prenom{}\normalsize{} 
          \vspace*{1mm}
      }}
\end{center}


\begin{center}
  \bf\em Les réponses aux questions sont à donner exclusivement sur cette feuille :
  les réponses données sur les feuilles précédentes ne seront pas prises en compte.
\end{center}


\AMCassociation{\id}

\formulaire

%\AMCaddpagesto{2}

}%exempalire
}%sujet

\begin{document}

%%%Options
\AMCrandomseed{1237893}

\def\AMCformQuestion#1{{\sc Question #1 :}}

\setdefaultgroupmode{withoutreplacement}
%%% Fin Options

%%% groupes

\element{general}{
  \begin{question}{prez}    
    Parmi les personnalités suivantes, laquelle a été présidente de la république française~?
    \begin{reponses}
      \bonne{René Coty}
      \mauvaise{Alain Prost}
      \mauvaise{Marcel Proust}
      \mauvaise{Claude Monet}
    \end{reponses}
  \end{question}
}

\element{general}{
  \begin{questionmult}{pref}    
    Parmi les villes suivantes, lesquelles sont des préfectures~?
    \begin{reponses}
      \bonne{Poitiers}
      \mauvaise{Sainte-Menehould}
      \bonne{Avignon}
    \end{reponses}
  \end{questionmult}
}

\element{general}{
  \begin{question}{nb-ue}
    Combien d'états sont membres de l'Union Européenne en janvier 2009~?
    \begin{reponseshoriz}[o]
      \mauvaise{15}
      \mauvaise{21}
      \mauvaise{25}
      \bonne{27}
      \mauvaise{31}
    \end{reponseshoriz}
  \end{question}
}

%%%% fin des groupes

\csvreader[head to column names]{liste.csv}{}{\sujet}

\end{document}
